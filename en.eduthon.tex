\documentclass{42-en}
\newcommand\qdsh{\texttt{42sh}}

\begin{document}
\title{Dash - ft\_yes}
\subtitle{yes, please}

\version{draft 2}

\summary
{
	YES, we are going to make \textbf{yes}
}

\maketitle

\tableofcontents


\chapter{Foreword}

Looks simple right?

\begin{42ccode}
#include <stdio.h>

int main()
{
	for(;;)
		printf("y\n");
}
\end{42ccode}

\info{
Harder, Better, Faster, Stronger
}

\chapter{Objective}
Create the fastest \textbf{\textit{/usr/bin/yes}}.\\\\\\
You can ignore interactive script easily with \textit{yes}.
\begin{42console}
	$ yes | rm -r large_directory
\end{42console}
\info{
	\url{https://en.wikipedia.org/wiki/Yes\_(Unix)}
}

\chapter{Instructions}

	\begin{itemize}

		\item Your program should never leak or unexpectedly quit(Segfault for example).
		\item If your program doesn’t compile, it’s a 0.
		\item We won't use norminette for dash C project.
		\item Evaluation will be done on 42 Seoul's Mac.
		\item This dash is a solo project.
		\item Turn in your code inside the turn-in repository.

	\end{itemize}


\newpage

\chapter{Exercice 00 : BUFFFFFFFFFFFFFFFFFFFFFFFFFFFFFFF}

\extitle{ft\_yes}
\exnumber{00}
\exfiles{ft\_yes.c}
\exforbidden{vmsplice, splice}
\makeheaderfilesforbidden
\begin{itemize}

	\item All other standard libraries and functions are allowed.
	\item You don't have to get arguments.
	\item Program should be at least faster than example code above. Otherwise it's a 0.
	\item Any assembly code is forbidden.
	\begin{42console}
		$ gcc -fno-asm -o ft_yes ft_yes.c
		$ ./ft_yes | pv >/dev/null
		136GiB 0:00:33 [4.60GiB/s] [    <=>                                            ]
	\end{42console}

\end{itemize}

\hint{
	\url{https://en.wikipedia.org/wiki/GNU_Core_Utilities}
}
\info{
	printf(), puts(), write()
}

\end{document}
